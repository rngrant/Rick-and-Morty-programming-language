%-----------------------------------------------------------------------------
%
%               Template for sigplanconf LaTeX Class
%
% Name:         sigplanconf-template.tex
%
% Purpose:      A template for sigplanconf.cls, which is a LaTeX 2e class
%               file for SIGPLAN conference proceedings.
%
% Guide:        Refer to "Author's Guide to the ACM SIGPLAN Class,"
%               sigplanconf-guide.pdf
%
% Author:       Paul C. Anagnostopoulos
%               Windfall Software
%               978 371-2316
%               paul@windfall.com
%
% Created:      15 February 2005
%
%-----------------------------------------------------------------------------


\documentclass[preprint]{sigplanconf}

% The following \documentclass options may be useful:

% preprint      Remove this option only once the paper is in final form.
% 10pt          To set in 10-point type instead of 9-point.
% 11pt          To set in 11-point type instead of 9-point.
% numbers       To obtain numeric citation style instead of author/year.

\usepackage{amsmath}

\newcommand{\cL}{{\cal L}}

\begin{document}

\special{papersize=8.5in,11in}
\setlength{\pdfpageheight}{\paperheight}
\setlength{\pdfpagewidth}{\paperwidth}

\conferenceinfo{CONF 'yy}{Month d--d, 20yy, City, ST, Country}
\copyrightyear{20yy}
\copyrightdata{978-1-nnnn-nnnn-n/yy/mm}
\copyrightdoi{nnnnnnn.nnnnnnn}

% Uncomment the publication rights you want to use.
%\publicationrights{transferred}
%\publicationrights{licensed}     % this is the default
%\publicationrights{author-pays}

\titlebanner{banner above paper title}        % These are ignored unless
\preprintfooter{short description of paper}   % 'preprint' option specified.

\title{An Esoteric Programming Language to Simulate Parallel Computing}
\subtitle{Contact: [email]@grinnell.edu}

\authorinfo{Reilly Grant}
           {Grinnell College}
           {[grantrei]}
\authorinfo{Tristan Knoth}
           {Grinnell College}
           {[knothtri17]}
\authorinfo{Chris Kottke}
           {Grinnell College}
           {[kottkech17]}

\maketitle

\begin{abstract}
Parallel computing is an important way to process over very large problems. However, developing parallel applications is extremely difficult. We develop Generalized Rick And Morty ProgrAmming (GRAMPA), an esoteric imperative programming language supporting a simple forking model in order to simulate parallelism and introduce students shared memory and other rudimentary parallel computing concepts. The language supports only a limited set of commands, and includes syntax is based on the popular cartoon Rick and Morty in order to present parallel computing concepts in a fun and accessible manner.
\end{abstract}

\section{Introduction}

As parallelism becomes the most important paradigm for large-scale information processing, developing parallel applications is becoming an increasingly important skill for any developer. Students who begin to think about splitting problems up between processors earlier in their computer science education will potentially see more success learning more sophisticated parallelism paradigms later on. To this end, the popular TV show Rick and Morty presents the perfect medium through which to introduce students to rudimentary parallelism concepts. We develop GRAMPA, a simple Turing Complete imperative language with syntax based on references to Rick and Morty that supports a simple model of forking across shared memory. In Rick and Morty, the main characters travel between dimensions. The show's clear conceptual connections to multithreading may help alleviate the pain of learning to parallelize simple algorithms. 

\section{Prior Work}

GRAMPA relies on a number of existing technologies, particularly the Haskell Parsec library. 

\section{GRAMPA Basics}

\section{Parallelism in GRAMPA}

\appendix
\section{Appendix Title}

This is the text of the appendix, if you need one.

\acks

Acknowledgments, if needed.

% We recommend abbrvnat bibliography style.

\bibliographystyle{abbrvnat}

% The bibliography should be embedded for final submission.

\begin{thebibliography}{}
\softraggedright

\bibitem[Smith et~al.(2009)Smith, Jones]{smith02}
P. Q. Smith, and X. Y. Jones. ...reference text...

\end{thebibliography}


\end{document}
