%-----------------------------------------------------------------------------
%
%               Template for sigplanconf LaTeX Class
%
% Name:         sigplanconf-template.tex
%
% Purpose:      A template for sigplanconf.cls, which is a LaTeX 2e class
%               file for SIGPLAN conference proceedings.
%
% Guide:        Refer to "Author's Guide to the ACM SIGPLAN Class,"
%               sigplanconf-guide.pdf
%
% Author:       Paul C. Anagnostopoulos
%               Windfall Software
%               978 371-2316
%               paul@windfall.com
%
% Created:      15 February 2005
%
%-----------------------------------------------------------------------------


\documentclass[numbers]{sigplanconf}

% The following \documentclass options may be useful:

% preprint      Remove this option only once the paper is in final form.
% 10pt          To set in 10-point type instead of 9-point.
% 11pt          To set in 11-point type instead of 9-point.
% numbers       To obtain numeric citation style instead of author/year.

\usepackage{amsmath}
\usepackage{enumitem}
\PassOptionsToPackage{hyphens}{url}\usepackage{hyperref}

\newcommand{\cL}{{\cal L}}

\begin{document}

\special{papersize=8.5in,11in}
\setlength{\pdfpageheight}{\paperheight}
\setlength{\pdfpagewidth}{\paperwidth}

\conferenceinfo{CONF 'yy}{Month d--d, 20yy, City, ST, Country}
\copyrightyear{20yy}
\copyrightdata{978-1-nnnn-nnnn-n/yy/mm}
\copyrightdoi{nnnnnnn.nnnnnnn}

% Uncomment the publication rights you want to use.
%\publicationrights{transferred}
%\publicationrights{licensed}     % this is the default
%\publicationrights{author-pays}

\titlebanner{}        % These are ignored unless
\preprintfooter{}   % 'preprint' option specified.

\title{Linear Temporal programming in an Esoteric Programming Language}
\subtitle{Contact: [email]@grinnell.edu}


\authorinfo{Tristan Knoth}
           {Grinnell College}
           {[knothtri17]}
\authorinfo{Reilly Grant}
           {Grinnell College}
           {[grantrei]}
\authorinfo{Chris Kottke}
           {Grinnell College}
           {[kottkech17]}

%remove ACM copyright stuff
\makeatletter
\def\@copyrightspace{\relax}
\makeatother
%

\maketitle

\begin{abstract}
This project attempts to create an esoteric programming language
deigned based on the popular cartoon, Rick and Morty. This programing
language will make use of integrated Linear Temporal logic to ensure
correctness, and advanced multithreading. In addition, this language
is intended to be interseting enough to attract students and
programmers unfamiliar with these concepts under the veneer of an
interesting language and thus introduce possibly non acadmemically oriented students
 to the concepts of temporal logic, and multithreading.
\end{abstract}

%Unsure what the purpose of this is
%\category{CR-number}{subcategory}{third-level}

%\keywords
%keyword1, keyword2

\section{Introduction}
The popular TV show Rick and Morty often uses the concepts of
Parallel timlines and alternative universes


\section{Prior Work}

\section{Proposed Work}
Temporal logic can be an extremely useful concept in various computing problems, particularly related to program verification. In order to ensure, for example, that a certain state is eventually reached, we need temporal logic. Similarly, we can use temporal logic to ensure that certain eventualities are never reached. A final important application is to ensure "fairness" in multi-threaded systems. A program may, for example, want to ensure that if a process makes a certain request often enough, that request is eventually fulfilled.

Thus, we propose to use Haskell to develop a simple Turing-complete computer language featuring temporal logic and some simple multi-threading utilities. Without multi-threading, we cannot use the temporal logic capabilities to their full potential. GHC already supports implicit parallelism, but 
\section{Timeline}


%\acks
%Acknowledgments, if needed.

% We recommend abbrvnat bibliography style.

\bibliographystyle{abbrvnat}

% The bibliography should be embedded for final submission.

\bibliography{sigplanconf}{}


\end{document}
